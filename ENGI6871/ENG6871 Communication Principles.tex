\documentclass[11pt]{article}
    
    \usepackage{caption}
    \usepackage{graphicx}
    \usepackage{mathtools}
    \usepackage{bookmark}
    \graphicspath{ {img/} }
    \setlength{\parindent}{0pt}
    \DeclareCaptionType{equ}[][]
    \usepackage[svgnames]{xcolor}
    
    \newcommand*{\plogo}{\fbox{$\mathcal{BM}$}}
    
    \usepackage{PTSerif}
    
    \begin{document} 
        
    \begin{titlepage}
    
        \raggedleft
        
        \vspace*{\baselineskip}
        
        {\Large Bryan Melanson}
        
        \vspace*{0.167\textheight}
        
        \textbf{\LARGE How to Not Fail}\\[\baselineskip]
        
        {\textcolor{Red}{\Huge Communication Principles}}\\[\baselineskip]
        
        {\Large \textit{While never going to class}}
        
        \vfill
        
        {\large Computer Engineering 2020 ~~\plogo}
        
        \vspace*{3\baselineskip}
    
    \end{titlepage}

    \pagebreak
    
%%%%%%%%%%%%%%%%%%%%%%%%%%%%%%%%%%%%%%%%%%%%%%%%%
    \pdfbookmark[section]{\contentsname}{toc}
    
    \tableofcontents
%%%%%%%%%%%%%%%%%%%%%%%%%%%%%%%%%%%%%%%%%%%%%%%%%
\pagebreak

\section{Signals and Signal Space}

\subsection{Energy}

\textbf{Signal Energy} $E_g$ is the energy that the voltage of $g(t)$ dissipates on a resistor. From circuits, we know that $E = V^2/R$, so for a time based signal with no theoretical $R$:

$$E_g = \int^{\infty}_{-\infty}|g(t)|^2 dt$$

\subsection{Power}

\textbf{Signal Power} is the measure of a signal where signal energy isn't finite. For a signal that doesn't go to zero as $t \rightarrow \infty$, a time average of the energy $P_g$ can be taken over a period $T$:

$$P_g = \lim_{t\to\infty} \frac{1}{T}\int_{T/2}^{-T/2} |g(t)|^2 dt$$

\hfill \break 
The square of $P_g$ is the \textbf{Root Mean Square} of value $g(t)$.

\subsection{Energy and Power Signals}

A signal with finite energy is an \textbf{Energy Signal}, and an signal with finite power is a \textbf{Power Signal}. In other terms, if $E_g < \infty$, or $0 < P_g < \infty$.

\subsection{Unit Impulse Signal}

$$\delta(t) = 0, t \neq 0$$
$$\int^{\infty}_{-\infty}\delta(t)dt = 1$$
\hfill \break 
A rectangular pulse of width $\epsilon$ and height $1/\epsilon$, the \textbf{Unit Impulse} is 0 everywhere but at $t = 0$.

$$\phi(t)\delta(t) = \phi(0)\delta(t)$$
$$\phi(t)\delta(t - T) = \phi(T)\delta(t - T)$$

\hfill \break
These equations follow from the understanding that $\delta(t)$ is the only point where the impulse has a value, and likewise when shifted $T$ units. The resulting equation will only be equal to the function $\phi$ at this specific point, multiplied by $\delta$ at this point.

\subsubsection{Sampling Property}

From the above, it can be seen that:

$$\int_{-\infty}^{\infty}\phi(t)\delta(t - T) = \phi(T)\int_{-\infty}^{\infty}\delta(t - T) dt = \phi(T)$$

\hfill \break
The $\phi(T)$ value can be shifted out of the $dt$ integration as a constant, leaving us with the result. This is known as the \textbf{Sampling} or \textbf{Sifting Property}. 

\end{document}