\documentclass[11pt]{article}
    
    \usepackage{caption}
    \usepackage{graphicx}
    \usepackage{mathtools}
    \usepackage{bookmark}
    \graphicspath{ {img/} }
    \setlength{\parindent}{0pt}
    \DeclareCaptionType{equ}[][]
    \usepackage[svgnames]{xcolor}
    
    \newcommand*{\plogo}{\fbox{$\mathcal{BM}$}}
    
    \usepackage{PTSerif}
    
    \begin{document} 
        
    \begin{titlepage}
    
        \raggedleft
        
        \vspace*{\baselineskip}
        
        {\Large Bryan Melanson}
        
        \vspace*{0.167\textheight}
        
        \textbf{\LARGE How to Not Fail}\\[\baselineskip]
        
        {\textcolor{Red}{\Huge Computer Architecture}}\\[\baselineskip]
        
        {\Large \textit{While never going to class}}
        
        \vfill
        
        {\large Computer Engineering 2020 ~~\plogo}
        
        \vspace*{3\baselineskip}
    
    \end{titlepage}

    \pagebreak
    
%%%%%%%%%%%%%%%%%%%%%%%%%%%%%%%%%%%%%%%%%%%%%%%%%
    \pdfbookmark[section]{\contentsname}{toc}
    
    \tableofcontents

%%%%%%%%%%%%%%%%%%%%%%%%%%%%%%%%%%%%%%%%%%%%%%%%%

\section{Instruction Set Architecture}

\subsection{Die Yield}

        
\begin{center}
    \begin{equation}
    \text{Die Yield} = (1 + (\frac{\text{Defects/Area} \cdot \text{Area of Die}}{N}))^{-N}
    \end{equation}
\end{center}

\textbf{Die Yield} is a value which defines how likely it is that a die will be non-defective during the fabrication process. This value is a probability $< 1$, and the probability that a die will be defective is $1 - $Die Yield. \\

When converting between units of area (mm$^2$, cm$^2$), be sure to repeat conversion between units twice.

\begin{center}
    1 mm$^2$ = $\frac{1}{10 \text{cm}} \cdot \frac{1}{10 \text{cm}} = \frac{1}{100}$ cm$^2$
\end{center}

\subsection{Wafer Yield}

\textbf{Wafer Yield} is a value which describes how many dies (chips) can be created from a single wafer. This calculation is done using the size of a chip, and the diameter of each wafer. It takes into effect the square or rectangle shape of a chip and losses due to rounded sides of a wafer.

\begin{center}
\begin{equation} \text{Dies per wafer} = 
    \frac{\pi \cdot (\text{Wafer Diameter}/2)^2}{\text{Die area}} - \frac{\pi \cdot \text{Wafer diameter}}{\sqrt{2 \cdot \text{Die area}}}    
\end{equation}
\end{center}

\end{document}
