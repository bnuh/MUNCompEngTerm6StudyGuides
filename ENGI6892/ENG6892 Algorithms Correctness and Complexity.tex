\documentclass[11pt]{article}
    
    \usepackage{caption}
    \usepackage{graphicx}
    \usepackage{mathtools}
    \usepackage{bookmark}
    \graphicspath{ {img/} }
    \setlength{\parindent}{0pt}
    \DeclareCaptionType{equ}[][]
    \usepackage[svgnames]{xcolor}
    
    \newcommand*{\plogo}{\fbox{$\mathcal{BM}$}}
    
    \usepackage{PTSerif}
    
    \begin{document} 
        
    \begin{titlepage}
    
        \raggedleft
        
        \vspace*{\baselineskip}
        
        {\Large Bryan Melanson}
        
        \vspace*{0.167\textheight}
        
        \textbf{\LARGE How to Not Fail}\\[\baselineskip]
        
        {\textcolor{Red}{\Huge Algorithm Correctness and Complexity}}\\[\baselineskip]
        
        {\Large \textit{While never going to class}}
        
        \vfill
        
        {\large Computer Engineering 2020 ~~\plogo}
        
        \vspace*{3\baselineskip}
    
    \end{titlepage}

    \pagebreak
    
%%%%%%%%%%%%%%%%%%%%%%%%%%%%%%%%%%%%%%%%%%%%%%%%%
    \pdfbookmark[section]{\contentsname}{toc}
    
    \tableofcontents
%%%%%%%%%%%%%%%%%%%%%%%%%%%%%%%%%%%%%%%%%%%%%%%%%

\section{Proof Outline Logic}

\subsection{Assertions}
An assertion is a \textbf{condition} that is expected to be true every time execution passes a particular point in a program. Available at run time in C, C++ and Java, an assertion which proves to not be true will prevent the program from running. \\

Assertions are valuable from a documentation standpoint, and for testing.

\subsection{Substitutions}

Replacing a free (ie global, not restricted to a certain scope) variable follows the syntax $P[x:E]$ where all occurences of $x$ are replaced by the expression $E$. This can be extended to multiple variables as $[x,y:z,x]$.

\subsection{Contracts}

A condition might be required before other conditions become valid: \\

953$I$ $\leq$ $V$ $\leq$ 1050$I$, provided 0 $\leq$ $V$ $\leq$ 10 \\\textbf{OR} 
[953$I$ $\leq$ $V$ $\leq$ 1050$I$, 0 $\leq$ $V$ $\leq$ 10] 

A contract [$y$ = 5, $x$ = 5] can be represented as: \\

\{$y$ = 4\} \\
$x$ := $y$ + 1 \\
\{$x$ = 5\} \\

This is a \textbf{Hoare Triple}, made up of a precondition, a command and a postcondition. This statement can be considered \textbf{Partially Correct} if for any values, when the precondition is satisfied, and the command is executed, it can only end in a state satisfying the postcondition.

\subsection{Proof Outlines}

A \textbf{Proof Outline} is a command annotated with assertions. It represents the summary of a proof, if correct. This can be proven to be correct using partial correctness. In sequence, if \{$P$\} $S$ \{$Q$\} $T$ \{$R$\}, and both \{$P$\} $S$ \{$Q$\} and \{$Q$\} $T$ \{$R$\} can be proven to be partially correct, then the full statement is a partially correct outline. \\

In a proof outline, each command will be preceded by an assertion. This is the \textbf{precondition}.

\subsubsection{Two Tailed If Rule}
\{$P$\} if ($E$) \{$Q$\} else \{$Q$\} $T$ \{$R$\} relies on the if/else conditions being provable along with the equivalent $E$. \\
\subsubsection{One Tailed If Rule}
\{$P$\} if ($E$) \{$Q$\} $T$ \{$R$\} relies on the if conditions being provable with $E$, and $\neg$$E$ and $R$ being provable.  \\

The initial precondition of a loop $P$ is known as the \textbf{Invariant}. It must be proven true at the start of each iteration of a loop, as well as when the loop terminates. \\

\end{document}
